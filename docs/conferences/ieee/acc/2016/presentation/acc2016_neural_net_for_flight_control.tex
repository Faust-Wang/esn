\documentclass{beamer}

\usetheme{Berkeley}

% Title Page 
\title[ACC2016]
{A Recurrent Neural Network for Non-Linear Control of a Fixed-Wing UAV}

\author
{Ohanes Dadian\inst{1} and Subodh Bhandari\inst{2}}

\institute[University]
{
	\inst{1}
	Department of Computer Science\\
	\inst{2}
	Department of Aerospace Engineering\\
	Cal Poly Pomona
}

\date[Conference Date]
{American Control Conference, 2016}
\subject{Flight Control}

\begin{document}
    % Title page
	\frame{\titlepage}
	\begin{frame}
		\frametitle{Topics}
		\tableofcontents[currentsection]
	\end{frame}
	
	\begin{frame}
		\section[Research Objectives]{Research Objectives}
		\frametitle{Research Objectives}
		\begin{itemize}
			\item Develop Neural Networks for Performing Non-Linear Control of Unmanned Aerial Vehicles.
			\begin{itemize}
				\item Both Feed Forward and Recurrent Models
			\end{itemize}
			\item A Recurrent Model will allow for closer aherence to control laws.
			\begin{itemize}
				\item Provide a feedback loop, allowing for adaptive control.
			\end{itemize}
			\item Provide an open platform for on campus UAV fleet.
		\end{itemize}
	\end{frame}
	
	\begin{frame}
		\section[Motivation]{Motivation}
		\frametitle{Motivation}
		\begin{itemize}
			\item Unmanned Aerial Vehicles (UAVs) requires the following to be registered in the National Airspace System (NAS).
			\begin{itemize}
				\item Further autonomy.
				\item Expand flight envelope.
			\end{itemize}
			\item Gain scheduling is a popular means of accomplishing this goal.
			\begin{itemize}
				\item Computationally Expensive.
				\item High Design and Implementation Costs.
				\item Rapid changes in gaings lead to instability.
				\item Pre-computed predictions may lead to vehicle failure.
			\end{itemize}
			\item A non-linear model would be better suited for this task.
			\begin{itemize}
				\item Neural Networks provide a connectionist and non-linear approach to solving computational problems.
				\item Solves problems the way humans do.
			\end{itemize}
		\end{itemize}
	\end{frame}
	
	\begin{frame}
		\section[Artificial Neural Networks]{Artificial Neural Networks}
		\frametitle{Artificial Neural Networks}
		\begin{itemize}
			\item Composed of several interconnected artificial neuraons that work in parallel to solve a problem.
			\item The artificial neuron is the corner stone of a neural network.
			\begin{itemize}
				\item Behaves like a biological nervous system.
				\begin{itemize}
					\item Neuron either fires or doesn't (1 or 0).
				\end{itemize}
				\item Multiplexes input signals into single output signal.
			\end{itemize}
		\end{itemize}
	\end{frame}
	
	\begin{frame}
		\frametitle{Artifical Neural Networks 2}
		\begin{itemize}
			\item Connections between neurons are weighted.
			\begin{itemize}
				\item Dictate the cost value of each possible path.
			\end{itemize}
			\item Weights of connections provide ability to find most fitted models.
			\begin{itemize}
				\item Follow path to least cost.
			\end{itemize}
			\item Neural Networks are trained using a learning algorithm.
			\begin{itemize}
				\item Uses training data which labels acceptable and unacceptable output.
			\end{itemize}
			\item Mean Square Error used to determine if the model is fitted.
		\end{itemize}	
	\end{frame}

	\begin{frame}
		\frametitle{Artifical Neural Networks 3}
		\begin{itemize}
			\item Neural Nets can be used to map complex relationships between inputs and outputs or perform pattern recognition.
			\item Adaptive nature allows for response to unexpected inputs and patterns.
			\item Come in many forms.
			\begin{itemize}
				\item Feed-forward
				\item Recurrent
			\end{itemize}
			\item Architecture
			\begin{itemize}
				\item Contain input layer, hidden layers, and output layer.
			\end{itemize}
		\end{itemize}	
	\end{frame}

	\begin{frame}
		\frametitle{Recurrent Neural Network}
		\begin{itemize}
			\item A recurrent network is a class of network where all connections form a directed cycle.
		\end{itemize}
	\end{frame}

	\begin{frame}
		\section[Echo State Network]{Echo State Network}
		\frametitle{Echo State Network}
		\begin{itemize}
			\item Provide current knowlede with simpler implementation.
			\item Random sparsely connected hidden layer.
			\item Weights are fixed.
			\item Hidden layer behaves as a reservior.
			\begin{itemize}
				\item Implements a leak rate for current knowledge.
			\end{itemize}
		\end{itemize}
	\end{frame}

	\begin{frame}
		\frametitle{ESN Algorithm}
		\begin{itemize}
			\item Generate a large random reservoir.
			\item Train using PID input and collect corresponding reservior activation states.
			\begin{itemize}
				\item Nonlinear transform of input signal (square wave) and the output signal (sine wave).
			\end{itemize}
			\item Compute the linear readout weights from the reservior using linear regression (Weiner-Hopf method).
			\item Compute Mean Square Error.
			\item Retrain if MSE is too high.
		\end{itemize}
	\end{frame}

	\begin{frame}
		\frametitle{Why ESN for Control and Identification}
		\begin{itemize}
			\item Simple design and implementation.
			\item{Can easily map linear systems methods to nonlinear systems.}
			\begin{itemize}
				\item{Nonlinear transform of input signal (square wave) and the output signal (sine wave).}
			\end{itemize}
			\item Short-term memory aids in enhancing learning process.
			\item Compute Mean Square Error.
			\item Retrain if MSE is too high.
		\end{itemize}
	\end{frame}

	\begin{frame}
		\frametitle{Why ESN for Control and Identification}
		\begin{itemize}
			\item Simple design and implementation.
			\item Can easily map linear systems methods to nonlinear systems.
			\begin{itemize}
				\item Nonlinear transform of input signal (square wave) and the output signal (sine wave).
			\end{itemize}
			\item Short-term memory aids in enhancing learning process.
			\item Highly adaptive behavior. 
		\end{itemize}
	\end{frame}

	\begin{frame}
		\section[Results]{Results}
		\frametitle{Results}

		\begin{itemize}
			\item The recurrent model was trained with data obtained from flight tests with our TwinEngine.
			\item Used data containing row, pith, and yaw axes for foreward and hoving motion.
		\end{itemize}
	\end{frame}

	\begin{frame}
		\section[Conclusion]{Conclusion}
		\frametitle{Conclusion}
		\begin{itemize}
			\item ESN can successfully perform identification and nonlinear control of UAVs.
			\item Able to perfrom both offline and online learning.
			\item Offline training allows for inversion of feedback linearization to be computed.
			\item Simulation showed controller performance signicantly improved in both open and closed loop responses.
			\item ESN faciliated adaptiblity with more ease than a feed forward model.
			\item Able to decrease delta of MSE with less training iterations.
			\item Shorter burts and Lyapunov stable.	
		\end{itemize}
	\end{frame}
\end{document}

